%%%%%%%%%%%%%%%%%%%%%%%%%%%%%%%%%%%%%%%%%%%%%%%%%%%%%%%%%%%%%%%%%%%%%%%%%%%%%%%%
%2345678901234567890123456789012345678901234567890123456789012345678901234567890
%        1         2         3         4         5         6         7         8

\documentclass[letterpaper, 10 pt, conference]{ieeeconf}  % Comment this line out if you need a4paper

%\documentclass[a4paper, 10pt, conference]{ieeeconf}      % Use this line for a4 paper

\IEEEoverridecommandlockouts                              % This command is only needed if 
                                                          % you want to use the \thanks command

\overrideIEEEmargins                                      % Needed to meet printer requirements.

% See the \addtolength command later in the file to balance the column lengths
% on the last page of the document

% The following packages can be found on http:\\www.ctan.org
%\usepackage{graphics} % for pdf, bitmapped graphics files
%\usepackage{epsfig} % for postscript graphics files
%\usepackage{mathptmx} % assumes new font selection scheme installed
%\usepackage{times} % assumes new font selection scheme installed
%\usepackage{amssymb}  % assumes amsmath package installed
%\usepackage{multirow}
\usepackage{color,colortbl}
\usepackage{graphicx}
\usepackage{amsmath} % assumes amsmath package installed
\usepackage{subfigure}
%\usepackage{epsfig}
\usepackage{url}
%\usepackage[linesnumbered,ruled,vlined]{algorithm2e}

\newcommand{\etal}{et~al. }
\newcommand{\Kir}{K_{\text{IR}}}
\newcommand{\Krgb}{K_{\text{RGB}}}
\newcommand{\xir}{x_{\text{IR}}}
\newcommand{\xrgb}{x_{\text{RGB}}}
\newcommand{\Kdepth}{K_{\text{Depth}}}
\begin{document}

\title{\LARGE \bf
    Task Pipeline Specification and Scheduling
}

\author{John Schulman and Arjun Singh}

\maketitle
%\thispagestyle{empty}
%\pagestyle{empty}


%%%%%%%%%%%%%%%%%%%%%%%%%%%%%%%%%%%%%%%%%%%%%%%%%%%%%%%%%%%%%%%%%%%%%%%%%%%%%%%%
%\begin{abstract}
%
%In the past few years, significant progress has been made on the problem of
%instance recognition, in which a vision system must estimate the identity and
%pose of an object from a database of object instances. However, for many
%robotics purposes, such a system needs to be nearly perfect to work in the real
%world.  The state of the art in computer vision has rapidly advanced over the
%past decade largely due to the advent of fundamental image datasets such as
%MNIST, Caltech-101, PASCAL, and ImageNet.  However, these datasets are usually
%geared towards image retrieval tasks, as they often contain assorted collections
%of images from the web that do not enable pose recovery.  Furthermore, they do
%not examine what can actually be solved today by computer vision for robotic
%perception.  To address these issues, we present a high-quality,
%large-scale dataset of 3D object instances, with accurate calibration
%information for every image. We anticipate that ``solving'' this dataset will
%effectively remove many perception-related problems for mobile, sensing-based
%robots.
%
%While there exist several 3D datasets (e.g. Willow Garage, NYU V2,
%RGB-D Object Dataset, B3DO), they either only contain contain a small
%number of moderate-quality object models and/or scenes, lack some of the
%information a robot might have (such as calibration), or lack instance-level
%data.  We present four contributions: (1) a dataset of 100 objects (and
%growing), composed of 600 3D point clouds and 600 high-resolution (12 MP)
%images spanning all views of each object, (2) a bundle adjustment method for
%calibrating a multi-sensor system, (3) details of our data collection system,
%which collects all required data for a single object in under 6 minutes with
%minimal human effort, and (4) multiple software components, used to automate
%multi-sensor calibration and simplify the data collection process.
%
%\end{abstract}
\begin{abstract}
    Stuff
\end{abstract}

\section{Introduction}
% TODO John -- partial

\section{Related Work}
% TODO John

\section{Overview}
\section{Implementation}
\subsection{Task Specification}
\subsection{Pipeline Structure}
% TODO John -- partial

\subsection{Parallelization}

\subsection{Scheduling}
% TODO John

\section{Results}
% TODO John -- partial

\section{Conclusions}


%\addtolength{\textheight}{-1cm}  % This command serves to balance the column lengths
%                                  % on the last page of the document manually. It shortens
%                                  % the textheight of the last page by a suitable amount.
%                                  % This command does not take effect until the next page
%                                  % so it should come on the page before the last. Make
%                                  % sure that you do not shorten the textheight too much.

%%%%%%%%%%%%%%%%%%%%%%%%%%%%%%%%%%%%%%%%%%%%%%%%%%%%%%%%%%%%%%%%%%%%%%%%%%%%%%%%



%%%%%%%%%%%%%%%%%%%%%%%%%%%%%%%%%%%%%%%%%%%%%%%%%%%%%%%%%%%%%%%%%%%%%%%%%%%%%%%%



%%%%%%%%%%%%%%%%%%%%%%%%%%%%%%%%%%%%%%%%%%%%%%%%%%%%%%%%%%%%%%%%%%%%%%%%%%%%%%%%

%%%%%%%%%%%%%%%%%%%%%%%%%%%%%%%%%%%%%%%%%%%%%%%%%%%%%%%%%%%%%%%%%%%%%%%%%%%%%%%%

\bibliography{references}
\bibliographystyle{unsrt}

%%%%%%%%%%%%%%%%%%%%%%%%%%%%%%%%%%%%%%%%%%%%%%%%%%%%%%%%%%%%%%%%%%%%%%%%%%%%%%%%

\end{document}
